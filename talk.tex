\documentclass[compress]{beamer}

\usepackage{tikz}
\usepackage{charter}   % Use the serif font charter
\usepackage{listings}
\usepackage{tikz}
\usetikzlibrary{shapes,arrows}
\usepackage{scalefnt}

\usefonttheme{serif}

%\usetheme{Madrid}


\definecolor{kentblue}{cmyk}{1.0,0.58,0,0.21}
\definecolor{kentpink}{cmyk}{0,1.0,0.13,0.17}
\definecolor{gray80}{rgb}{0.2,0.2,0.2}
\definecolor{gray10}{rgb}{0.8, 0.8, 0.8}
\definecolor{mygreen}{rgb}{0,0.5,0}

\usebeamercolor[bg]{frametitle}
\setbeamercolor{frametitle}{bg=kentblue}
\setbeamercolor{frametitle}{fg=white}
\setbeamercolor{title}{fg=kentpink}
\setbeamercolor{block title}{fg=kentpink}
\setbeamercolor{section in head}{bg=black}
\setbeamercolor{palette secondary}{fg=white,bg=kentblue}
\setbeamercolor{palette primary}{bg=gray80,fg=white}
\setbeamercolor{palette tertiary}{bg=black,fg=white}
\setbeamercolor{palette quaternary}{bg=black,fg=white}


%\logo{\includegraphics[height=1cm]{kent-logo}}

%%%%%%%%%%%%%%%%%%%%%%%%%
\setbeamertemplate{block begin}{
  \vskip.75ex
  \begin{beamercolorbox}[ht=3.5ex,dp=0.5ex,center,leftskip=-1em,colsep*=.75ex]{block title}%
    \usebeamerfont*{block title}%
    {\phantom{Gg}\insertblocktitle}% phantom because of baseline problem
  \end{beamercolorbox}%
  {\ifbeamercolorempty[bg]{block body}{}{\nointerlineskip\vskip-0.5pt}}%
  \usebeamerfont{block body}%
  \begin{beamercolorbox}[leftskip=1em,colsep*=.75ex,sep=0.5ex,vmode]{block body}%
    \ifbeamercolorempty[bg]{block body}{\vskip-.25ex}{\vskip-.75ex}\vbox{}%
}
\setbeamertemplate{block end}{
  \end{beamercolorbox}
}

%\tikzstyle{prog} = [rectangle, text centered]
%\tikzstyle{block} = [rectangle, draw, text centered]
%\tikzstyle{line} = [draw, -latex']
%\tikzstyle{end} = [node distance=3cm, minimum height=2em]
%\tikzstyle{node} = [circle, draw, text centered, minimum width=.2cm]

\title{Research: Past and Present}
\author{Edd Barrett}
\date{\today}

\institute{%
	Software Development Team\\
	King's College London, England
}

\lstset{
  basicstyle=\ttfamily\footnotesize,
  breaklines=true,
  stringstyle=\ttfamily,
  framexleftmargin=1pt,
  frame=tb,
  backgroundcolor=\color{white}
}

\begin{document}

\begin{frame}
  \titlepage
\end{frame}

\section{Range Analysis of Binaries with Decision Procedures}

\begin{frame}
  \sectionpage
\end{frame}

\subsection{Reverse Engineering and Verification}

\begin{frame}[fragile]
	\frametitle{\insertsubsection}

	Two subjects of growing imprtance:
	\vfill
	\begin{block}{Reverse Engineering}
                The art of developing an understanding of a program from
                the compiled code alone
	\end{block}
	\vfill
	\begin{block}{Verification}
                Proving that a software product conforms to a set of
                safety, correctness and reliability criteria.
	\end{block}

\end{frame}


\subsection{Applications for Reverse Engineering}
\begin{frame}[fragile]
	\frametitle{\insertsubsection}

	\begin{itemize}
		\vfill
	\item Malware detection and classification.
		\vfill
	\item Loss of code.
		\vfill
	\item Penetration testing.
		\vfill
	\item Interfacing and interoperating.
		\vfill
	\end{itemize}

\end{frame}

\subsection{Applications for Verification}
\begin{frame}[fragile]
	\frametitle{\insertsubsection}

	\begin{itemize}
		\vfill
	\item Checking memory safety.
		\vfill
	\item Proving error state unreachability.
		\vfill
	\item Proving an upper bound on memory consumption.
		\vfill
	\item Proving worst case execution time.
		\vfill
	\end{itemize}

\end{frame}


\subsection{Why at the Binary Level}
\begin{frame}[fragile]
	\frametitle{\insertsubsection}

	But why work at the binary level?
	\vfill

	\begin{block}{Reverse Engineering}
		\begin{itemize}
		\item No access to source code.
		\item No choice.
		\end{itemize}
	\end{block}

	\vfill

	\begin{block}{Verification}
		\begin{itemize}
		\item Traditionally done at the source level.
		\item Good reasons for doing it at the binary level.
		\end{itemize}
	\end{block}

\end{frame}


\subsection{WYSINWYX}
\begin{frame}[fragile]
	\frametitle{\insertsubsection}

	\textbf{\Large W}hat \textbf{\Large Y}ou \textbf{\Large S}ee \textbf{\Large I}s \textbf{\Large N}ot \textbf{\Large W}hat \textbf{\Large Y}ou e\textbf{\Large X}ecute.
	\vfill

	\begin{itemize}
		\item It is not the source code that is executed\footnote{Forget scripting languages and VMs for now.}.
		\item Compilers are often ``too clever''.
		\item Compilers are also buggy.
	\end{itemize}
	\vfill

	\begin{lstlisting}
	// password no longer used
	memset(password, '\0', len);  // line optimised away
	free(password);
	\end{lstlisting}
	\vfill


\end{frame}

\subsection{The General Approach}
\begin{frame}[fragile]
	\frametitle{\insertsubsection}

        Often, reversing and verification amounts to collecting the
        potential values that may arise at each point in a program.
\end{frame}

\subsection{Concrete Interpretation}

\begin{frame}[fragile]
	\frametitle{\insertsubsection}

	\vspace{-1cm}
	\begin{center}
	\begin{minipage}{.5\textwidth}
	\begin{tabbing}
		1234\=1234\=1234\=\kill
		1:\>$x \leftarrow 6$\\
		2:\>while $x > 0$ do\\
		3:\>\>x $\leftarrow x - 2$\\
		4:\>done\\
		5:\>
	\end{tabbing}
	\end{minipage}\qquad\qquad~~
	\begin{minipage}{.5\textwidth}
		\tikzstyle{nnn}=[circle]
		\begin{tikzpicture}[auto,node distance=1.3cm, thick]
			\node[nnn] (n1) {1};
			\node[nnn] (n2) [below of=n1] {2};
			\node[nnn] (n3) [below of=n2] {3};
			\node[nnn] (n4) [below of=n3] {4};
			\node[nnn] (n5) [below left of=n2] {5};
			\draw [->] (n1) -- (n2);
			\draw [->] (n2) -- (n3);
			\draw [->] (n3) -- (n4);
			\draw [->] (n2) -- (n5);
			\draw [->] (n4) to [out=0, in=0] (n2);
		\end{tikzpicture}
	\end{minipage}
	\end{center}
	\vfill

	\begin{overlayarea}{\textwidth}{0cm}
		\vspace{-1cm}
	\only<1>{%
	\[
		\begin{array}{rcl}
		S_1 &=& \top\\
		S_2 &=& \{6\} \cup S_4\\
		S_3 &=& \{ x \mid x \in S_2 \wedge x > 0 \}\\
		S_4 &=& \{x - 2 \mid x \in S_3 \}\\
		S_5 &=& \{x \mid x \in S_2 \wedge x \leq 0\}
		\end{array}
	\]
	}
	\only<2>{%
	\[
	\begin{array}{rcl}
		S_1 &=& \top\\
		S_2 &=& \{6, 4, 2, 0\}\\
		S_3 &=& \{6, 4, 2\}\\
		S_4 &=& \{4, 2, 0\}\\
		S_5 &=& \{0\}\\
	\end{array}
	\]
	}
	\end{overlayarea}
	\vfill
\end{frame}

\subsection{Concrete Interpretation (2)}

\begin{frame}[fragile]
	\frametitle{\insertsubsection}

	\vspace{-1cm}
	\begin{center}
	\begin{minipage}{.5\textwidth}
	\begin{tabbing}
		1234\=1234\=1234\=\kill
		1:\>$x \leftarrow {\color{red}{66666}}$\\
		2:\>while $x > 0$ do\\
		3:\>\>x $\leftarrow x - 2$\\
		4:\>done\\
		5:\>
	\end{tabbing}
	\end{minipage}\qquad\qquad~~
	\begin{minipage}{.5\textwidth}
		\tikzstyle{nnn}=[circle]
		\begin{tikzpicture}[auto,node distance=1.3cm, thick]
			\node[nnn] (n1) {1};
			\node[nnn] (n2) [below of=n1] {2};
			\node[nnn] (n3) [below of=n2] {3};
			\node[nnn] (n4) [below of=n3] {4};
			\node[nnn] (n5) [below left of=n2] {5};
			\draw [->] (n1) -- (n2);
			\draw [->] (n2) -- (n3);
			\draw [->] (n3) -- (n4);
			\draw [->] (n2) -- (n5);
			\draw [->] (n4) to [out=0, in=0] (n2);
		\end{tikzpicture}
	\end{minipage}
	\end{center}
	\vfill

	\begin{overlayarea}{\textwidth}{0cm}
		\vspace{-1cm}
	\only<1>{%
	\[
		\begin{array}{rcl}
		S_1 &=& \top\\
			S_2 &=& \{{\color{red}{66666}}\} \cup S_4\\
		S_3 &=& \{ x \mid x \in S_2 \wedge x > 0 \}\\
		S_4 &=& \{x - 2 \mid x \in S_3 \}\\
		S_5 &=& \{x \mid x \in S_2 \wedge x \leq 0\}
		\end{array}
	\]
	}
	\only<2>{%
	\[
	\begin{array}{rcl}
		S_1 &=& \top\\
		S_2 &=& \{66666, 66664, \ldots, 2, 0\}\\
		S_3 &=& \{66666, 66664, \ldots, 4, 2\}\\
		S_4 &=& \{66664, 66662, \ldots, 2, 0\}\\
		S_5 &=& \{0\}\\
	\end{array}
	\]
	}
	\end{overlayarea}
	\vfill
\end{frame}

\subsection{Abstract Interpretation with Intervals}

\begin{frame}[fragile]
	\frametitle{\insertsubsection}

	\vspace{-1cm}
	\begin{center}
	\begin{minipage}{.5\textwidth}
	\begin{tabbing}
		1234\=1234\=1234\=\kill
		1:\>$x \leftarrow 66666$\\
		2:\>while $x > 0$ do\\
		3:\>\>x $\leftarrow x - 2$\\
		4:\>done\\
		5:\>
	\end{tabbing}
	\end{minipage}\qquad\qquad~~
	\begin{minipage}{.5\textwidth}
		\tikzstyle{nnn}=[circle]
		\begin{tikzpicture}[auto,node distance=1.3cm, thick]
			\node[nnn] (n1) {1};
			\node[nnn] (n2) [below of=n1] {2};
			\node[nnn] (n3) [below of=n2] {3};
			\node[nnn] (n4) [below of=n3] {4};
			\node[nnn] (n5) [below left of=n2] {5};
			\draw [->] (n1) -- (n2);
			\draw [->] (n2) -- (n3);
			\draw [->] (n3) -- (n4);
			\draw [->] (n2) -- (n5);
			\draw [->] (n4) to [out=0, in=0] (n2);
		\end{tikzpicture}
	\end{minipage}
	\end{center}
	\vfill

	\begin{overlayarea}{\textwidth}{0cm}
		\vspace{-1cm}
	\only<1>{%
	\[
		\begin{array}{rcl}
		S'_1 &=& \top\\
		S'_2 &=& [66666, 66666] \sqcup S'_4\\
		S'_3 &=& S'_2 \sqcap [1, +\infty]\\
		S'_4 &=& S'_3 - [2,2]\\
		S'_5 &=& S'_2 \sqcap [-\infty, 0]\\
		\end{array}
	\]
	}
	\only<2>{%
	\[
	\begin{array}{rcl}
		S_1 &=& \top\\
		S_2 &=& [-1, 66666]\\
		S_3 &=& [1, 66666]\\
		S_4 &=& [-1, 66664]\\
		S_5 &=& [-1, 0]
	\end{array}
	\]
	}
	\end{overlayarea}
	\vfill
\end{frame}


\subsection{So What's the Problem?}
\begin{frame}[fragile]
	\frametitle{\insertsubsection}

	\begin{itemize}
		\item The control flow graph (CFG) is required.
		\item Indirect jumps in binary code make the CFG hard to extract.
	\end{itemize}

	\vfill

\begin{lstlisting}
mov eax, [ebp-0x8]  ; eax := *(ebp - 8)
sub eax, 0x2        ; eax := eax - 2
cmp eax, 0x5        ; cf := (0 =< eax < 5)
                    ; zf := (eax = 5)
ja  0xd8            ; jump if cf = 0 and zf = 0
jmp [0x8048a0c + eax*4]
\end{lstlisting}
\vfill

	\begin{itemize}
	\item Need to know the possible values of \texttt{eax} at the \texttt{jmp}.
	\item Can't get the register values without the CFG.
	\item Can't get CFG without the register values.
	\item Need to consider bitwise details.
	\end{itemize}
\vfill
\end{frame}


\subsection{Determining Indirect Jump Addresses with SAT}
\begin{frame}[fragile]
	\frametitle{\insertsubsection}

	\begin{itemize}
	\item CPU instructions can be encoded as Boolean formulae.
	\end{itemize}
	\vfill
	E.g. \texttt{shl eax, 2}:

	\[
                \neg eax'_{0} \wedge \neg eax'_{1} \wedge (eax'_{2}
                \Leftrightarrow eax_{0}) \wedge \ldots \wedge (eax'_{31}
                \Leftrightarrow eax_{29})
	\]

	%Where $\vec{eax}$ and $\vec{eax'}$ represent \texttt{eax} prior to and after the shift.

	\vfill

	\begin{itemize}
	\item These formulae can be composed to model blocks of code.
	\end{itemize}

\end{frame}


\subsection{Determining Indirect Jump Addresses with SAT (2)}
\begin{frame}[fragile]
	\frametitle{\insertsubsection}

	Chapters 3 and 4 of my Thesis show that, given such a formula:

	\vfill

	\begin{itemize}
	\item A range of values can be derived for a bit vector of interest.
		\begin{itemize}
		\item E.g. Possible indirect jump addresses.
		\item Only $2n$ calls to a SAT solver, ($n$ is the width of the vector).
		\item Incremental SAT.
		\end{itemize}

		\vfill

	\item The range can be refined into a more precise set.
		\begin{itemize}
		\item Requires quantifier elimination.
		\end{itemize}
	\end{itemize}

	\vfill

	Using the above, indirect jumps can be resolved.

\end{frame}

\subsection{Slow Convergence}
\begin{frame}[fragile]
	\frametitle{\insertsubsection}

	\begin{itemize}
	\item Now we have the CFG.
		\begin{itemize}
		\item Abstract interpretation can proceed.
		\end{itemize}

	\vfill
	\item However, AI can suffer from slow convergence.
	%	\vfill
	%\item Consider the following loop:
	\end{itemize}

	\vfill

	\begin{center}
	\verb!for (x = 0; x < 9999; x++) { ... }!
	\end{center}

	\vfill
	%\texttt{x} inside the loop:
	\[
		\langle [0, 0], [0, 1], [0, 2], \ldots, [0, 9998], [0, 9999], [0, 9999]\rangle
	\]

	\vfill
	\begin{itemize}
	\item Solving will be too slow.
	\end{itemize}

\end{frame}


\subsection{Widening}
\begin{frame}[fragile]
	\frametitle{\insertsubsection}

	\begin{itemize}
	\item It is common to accelerate convergence with widening.
		\begin{itemize}
		\item If after $x$ iterations, convergence is not met, widen.
		\end{itemize}
	\end{itemize}

	\vfill

	\[
	\begin{aligned}
	\triangledown_I : I &\rightarrow I\\
	\bot_I \triangledown_I [c, d] &= [c, d]\\
	[a, b] \triangledown_I [c, d] &= [\text{if~} c < a \text{~then~} -\infty \text{~else~} a,\\
		&~~~\text{~if~} d >b \text{~then~} +\infty \text{~else~} b]
	\end{aligned}
	\]

	\vfill

	E.g. Widen after 3 iterations:
	\[
	\langle [0, 0], [0, 1], [0, 2], [0, +\infty], [0, +\infty]\rangle
	\]

	\vfill


	\begin{itemize}
	\item Imprecise.
	\end{itemize}

\end{frame}


\subsection{Range Analysis as an Optimisation Problem}
\begin{frame}[fragile]
	\frametitle{\insertsubsection}

	\begin{itemize}
	\item My approach is different.

	\vfill

	\item Instead of solving the abstract semantics to a FP, the abstract semantics are decomposed into a series of linear programs.

	\vfill

	\item This allows the solution to the abstract semantics to be found directly.

	\vfill

	\item No widening required, therefore precise.

	\end{itemize}

\end{frame}

\subsection{Range Analysis as an Optimisation Problem (2)}
\begin{frame}[fragile]
	\frametitle{\insertsubsection}

\begin{lstlisting}[showlines=true, numbers=left,escapechar=\!]
i := 10;
while (i !$\geq$! m)
        m := m + 1;
endwhile
// last program point
\end{lstlisting}

\vfill

\[
\begin{array}{rl}
S_1' =& \langle [-2^{31}, 2^{31}-1], [5, 20]\rangle\\
S_2'^{\star} =& \langle [10, 10], m\rangle~\text{where}~ \langle i, m\rangle = S_1'\\
S_2' =& S_2'^{\star} \sqcup S_4'\\
S_3' =& \langle [l_i, u_i] \sqcap [l_m, 2^{31}-1], [l_m, u_m] \sqcap [-2^{31}, u_i]\rangle\\
             &\qquad\text{where}~\langle [l_i, u_i], [l_m, u_m]\rangle = S_2'\\
S_4' =& \langle i, [min(l_m + 1, 2^{31}-1), min(u_m + 1, 2^{31} - 1)]\rangle\\
             &\qquad\text{where}~\langle i, [l_m, u_m]\rangle = S'_3\\
S_5' =& \langle [l_i, u_i] \sqcap [-2^{31}, u_m-1], [l_m, u_m] \sqcap [l_i+1, 2^{31}-1]\rangle\\
     &\qquad\text{where}~\langle [l_i, u_i], [l_m, u_m]\rangle = S_2'\\
\end{array}
\]

\end{frame}


\subsection{Range Analysis as an Optimisation Problem (3)}
\begin{frame}[fragile]
	\frametitle{\insertsubsection}

\begin{itemize}
\item Is decomposed into:
\end{itemize}

\[
\small
minimise\quad\sum_{j=1}^{5} (u_{i,j} - l_{i,j}) + \sum_{j=1}^{5} (u_{m,j} - l_{m,j})~s.t.
\]
\[
\footnotesize
\begin{array}{rcl@{\quad}c@{\quad}rcl@{\quad}clll}
l_{i,1} & = & -2^{31} & \wedge & u_{i,1} & = & 2^{31} - 1 & \wedge &  \\
l_{i,2{\star}} & = & 10 & \wedge & u_{i,2{\star}} & = & 10 & \wedge & \\
l_{i,2} & = & min(l_{i,2{\star}}, l_{i,4}) & \wedge & u_{i,2} & = & max(u_{i,2{\star}}, u_{i,4}) & \wedge \\
l_{i,3} & = & max(l_{i,2}, l_{m,2}) & \wedge & u_{i,3} & = & u_{i,2} & \wedge \\
l_{i,4} & = & l_{i,3} & \wedge & u_{i,4} & = & u_{i,3} & \wedge \\
l_{i,5} & = & l_{i,2} & \wedge & u_{i,5} & = & min(u_{i,2}, u_{m,2}-1) & \wedge \\
%\hline
l_{m,1} & = & 5 & \wedge & u_{m,1} & = & 20 & \wedge \\
l_{m,2{\star}} & = & l_{m,1} & \wedge & u_{m,2{\star}} & = & u_{m,1} & \wedge \\
l_{m,2} & = & min(l_{m,2{\star}}, l_{m,4}) & \wedge & u_{m,2} & = & max(u_{m,2{star}}, u_{m,4}) & \wedge \\
l_{m,3} & = & l_{m,2} & \wedge & u_{m,3} & = & min(u_{i,2}, u_{m,2}) & \wedge \\
l_{m,4} & = & min({l_{m,3}} + 1, 2^{31} - 1) & \wedge & u_{m,4} & = & min(u_{m,3} + 1, 2^{31} - 1) & \wedge \\
l_{m,5} & = & max(l_{m,2}, l_{i,2} + 1) & \wedge & u_{m,5} & = & u_{m,2}
\end{array}
\]

\end{frame}


\subsection{Range Analysis as an Optimisation Problem (4)}
\begin{frame}[fragile]
	\frametitle{\insertsubsection}

\begin{itemize}
\item The min/max terms are non convex.
\vfill
\item $z = min(x, y) \equiv z \leq x \wedge z \leq y \wedge (z = x \vee z = y)$.
\vfill
\item Use a binary search with heuristics to find the best solution.
\end{itemize}

\vfill

\[
\begin{array}{rcl}
S'_1 & = & [-2^{31}, 2^{31} - 1] \times [5, 20] \\
S'_2 & = & [10, 10] \times [5, 20] \\
S'_3 & = & [10, 10] \times [5, 10]
\end{array}
\quad
\begin{array}{rcl}
S'_4 & = & [10, 10] \times [6, 11] \\
S'_5 & = & [10, 10] \times [11, 20]
\end{array}
\]

\end{frame}

\section{And Next?}

\begin{frame}
  \sectionpage
\end{frame}


\subsection{Getting to grips with PyPy}
\begin{frame}[fragile]
	\frametitle{\insertsubsection}
\end{frame}

\end{document}
