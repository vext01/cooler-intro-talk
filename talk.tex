\documentclass[compress]{beamer}

\usepackage{tikz}
\usepackage{charter}   % Use the serif font charter
\usepackage{listings}
\usepackage{tikz}
\usetikzlibrary{shapes,arrows}
\usepackage{scalefnt}

\usefonttheme{serif}

%\usetheme{Madrid}


\definecolor{kentblue}{cmyk}{1.0,0.58,0,0.21}
\definecolor{kentpink}{cmyk}{0,1.0,0.13,0.17}
\definecolor{gray80}{rgb}{0.2,0.2,0.2}
\definecolor{gray10}{rgb}{0.8, 0.8, 0.8}
\definecolor{mygreen}{rgb}{0,0.5,0}

\usebeamercolor[bg]{frametitle}
\setbeamercolor{frametitle}{bg=kentblue}
\setbeamercolor{frametitle}{fg=white}
\setbeamercolor{title}{fg=kentpink}
\setbeamercolor{block title}{fg=kentpink}
\setbeamercolor{section in head}{bg=black}
\setbeamercolor{palette secondary}{fg=white,bg=kentblue}
\setbeamercolor{palette primary}{bg=gray80,fg=white}
\setbeamercolor{palette tertiary}{bg=black,fg=white}
\setbeamercolor{palette quaternary}{bg=black,fg=white}


%\logo{\includegraphics[height=1cm]{kent-logo}}

%%%%%%%%%%%%%%%%%%%%%%%%%
\setbeamertemplate{block begin}{
  \vskip.75ex
  \begin{beamercolorbox}[ht=3.5ex,dp=0.5ex,center,leftskip=-1em,colsep*=.75ex]{block title}%
    \usebeamerfont*{block title}%
    {\phantom{Gg}\insertblocktitle}% phantom because of baseline problem
  \end{beamercolorbox}%
  {\ifbeamercolorempty[bg]{block body}{}{\nointerlineskip\vskip-0.5pt}}%
  \usebeamerfont{block body}%
  \begin{beamercolorbox}[leftskip=1em,colsep*=.75ex,sep=0.5ex,vmode]{block body}%
    \ifbeamercolorempty[bg]{block body}{\vskip-.25ex}{\vskip-.75ex}\vbox{}%
}
\setbeamertemplate{block end}{
  \end{beamercolorbox}
}

%\tikzstyle{prog} = [rectangle, text centered]
%\tikzstyle{block} = [rectangle, draw, text centered]
%\tikzstyle{line} = [draw, -latex']
%\tikzstyle{end} = [node distance=3cm, minimum height=2em]
%\tikzstyle{node} = [circle, draw, text centered, minimum width=.2cm]

\title{Research: Past and Present}
\author{Edd Barrett}
\date{\today}

\institute{%
	Software Development Team\\
	King's College London, England
}

\lstset{
  basicstyle=\ttfamily\footnotesize,
  breaklines=true,
  stringstyle=\ttfamily,
  framexleftmargin=1pt,
  frame=tb,
  backgroundcolor=\color{white}
}

\begin{document}

\begin{frame}
  \titlepage
\end{frame}

\section{Past Research}

\begin{frame}
  \sectionpage
\end{frame}

\subsection{Reverse Engineering and Verification}

\begin{frame}[fragile]
	\frametitle{\insertsubsection}

	Two subjects of growing imprtance:
	\vfill
	\begin{block}{Reverse Engineering}
                The art of developing an understanding of a program from
                the compiled code alone
	\end{block}
	\vfill
	\begin{block}{Verification}
                Proving that a software product conforms to a set of
                safety, correctness and reliability criteria.
	\end{block}

\end{frame}


\subsection{Applications for Reverse Engineering}
\begin{frame}[fragile]
	\frametitle{\insertsubsection}

	\begin{itemize}
		\vfill
	\item Malware detection and classification.
		\vfill
	\item Loss of code.
		\vfill
	\item Penetration testing.
		\vfill
	\item Interfacing and interoperating.
		\vfill
	\end{itemize}

\end{frame}

\subsection{Applications for Verification}
\begin{frame}[fragile]
	\frametitle{\insertsubsection}

	\begin{itemize}
		\vfill
	\item Checking memory safety.
		\vfill
	\item Proving error state unreachability.
		\vfill
	\item Proving an upper bound on memory consumption.
		\vfill
	\item Proving worst case execution time.
		\vfill
	\end{itemize}

\end{frame}


\subsection{Why at the Binary Level}
\begin{frame}[fragile]
	\frametitle{\insertsubsection}

	But why work at the binary level?
	\vfill

	\begin{block}{Reverse Engineering}
		\begin{itemize}
		\item No access to source code.
		\item No choice.
		\end{itemize}
	\end{block}

	\vfill

	\begin{block}{Verification}
		\begin{itemize}
		\item Traditionally done at the source level.
		\item Good reasons for doing it at the binary level.
		\end{itemize}
	\end{block}

\end{frame}


\subsection{WYSINWYX}
\begin{frame}[fragile]
	\frametitle{\insertsubsection}

	\textbf{\Large W}hat \textbf{\Large Y}ou \textbf{\Large S}ee \textbf{\Large I}s \textbf{\Large N}ot \textbf{\Large W}hat \textbf{\Large Y}ou e\textbf{\Large X}ecute.
	\vfill

	\begin{itemize}
		\item It is not the source code that is executed\footnote{Forget scripting languages and VMs for now.}.
		\item Compilers are often ``too clever''.
		\item Compilers are also buggy.
	\end{itemize}
	\vfill

	\begin{lstlisting}
	// password no longer used
	memset(password, '\0', len);  // line optimised away
	free(password);
	\end{lstlisting}
	\vfill


\end{frame}

\subsection{The General Approach}
\begin{frame}[fragile]
	\frametitle{\insertsubsection}

        Often, reversing and verification amounts to collecting the
        potential values that may arise at each point in a program.
\end{frame}

\subsection{Concrete Interpretation}

\begin{frame}[fragile]
	\frametitle{\insertsubsection}

	\vspace{-1cm}
	\begin{center}
	\begin{minipage}{.5\textwidth}
	\begin{tabbing}
		1234\=1234\=1234\=\kill
		1:\>$x \leftarrow 6$\\
		2:\>while $x > 0$ do\\
		3:\>\>x $\leftarrow x - 2$\\
		4:\>done\\
		5:\>
	\end{tabbing}
	\end{minipage}\qquad\qquad~~
	\begin{minipage}{.5\textwidth}
		\tikzstyle{nnn}=[circle]
		\begin{tikzpicture}[auto,node distance=1.3cm, thick]
			\node[nnn] (n1) {1};
			\node[nnn] (n2) [below of=n1] {2};
			\node[nnn] (n3) [below of=n2] {3};
			\node[nnn] (n4) [below of=n3] {4};
			\node[nnn] (n5) [below left of=n2] {5};
			\draw [->] (n1) -- (n2);
			\draw [->] (n2) -- (n3);
			\draw [->] (n3) -- (n4);
			\draw [->] (n2) -- (n5);
			\draw [->] (n4) to [out=0, in=0] (n2);
		\end{tikzpicture}
	\end{minipage}
	\end{center}
	\vfill

	\begin{overlayarea}{\textwidth}{0cm}
		\vspace{-1cm}
	\only<1>{%
	\[
		\begin{array}{rcl}
		S_1 &=& \top\\
		S_2 &=& \{6\} \cup S_4\\
		S_3 &=& \{ x \mid x \in S_2 \wedge x > 0 \}\\
		S_4 &=& \{x - 2 \mid x \in S_3 \}\\
		S_5 &=& \{x \mid x \in S_2 \wedge x \leq 0\}
		\end{array}
	\]
	}
	\only<2>{%
	\[
	\begin{array}{rcl}
		S_1 &=& \top\\
		S_2 &=& \{6, 4, 2, 0\}\\
		S_3 &=& \{6, 4, 2\}\\
		S_4 &=& \{4, 2, 0\}\\
		S_5 &=& \{0\}\\
	\end{array}
	\]
	}
	\end{overlayarea}
	\vfill
\end{frame}

\subsection{Concrete Interpretation (2)}

\begin{frame}[fragile]
	\frametitle{\insertsubsection}

	\vspace{-1cm}
	\begin{center}
	\begin{minipage}{.5\textwidth}
	\begin{tabbing}
		1234\=1234\=1234\=\kill
		1:\>$x \leftarrow {\color{red}{66666}}$\\
		2:\>while $x > 0$ do\\
		3:\>\>x $\leftarrow x - 2$\\
		4:\>done\\
		5:\>
	\end{tabbing}
	\end{minipage}\qquad\qquad~~
	\begin{minipage}{.5\textwidth}
		\tikzstyle{nnn}=[circle]
		\begin{tikzpicture}[auto,node distance=1.3cm, thick]
			\node[nnn] (n1) {1};
			\node[nnn] (n2) [below of=n1] {2};
			\node[nnn] (n3) [below of=n2] {3};
			\node[nnn] (n4) [below of=n3] {4};
			\node[nnn] (n5) [below left of=n2] {5};
			\draw [->] (n1) -- (n2);
			\draw [->] (n2) -- (n3);
			\draw [->] (n3) -- (n4);
			\draw [->] (n2) -- (n5);
			\draw [->] (n4) to [out=0, in=0] (n2);
		\end{tikzpicture}
	\end{minipage}
	\end{center}
	\vfill

	\begin{overlayarea}{\textwidth}{0cm}
		\vspace{-1cm}
	\only<1>{%
	\[
		\begin{array}{rcl}
		S_1 &=& \top\\
			S_2 &=& \{{\color{red}{66666}}\} \cup S_4\\
		S_3 &=& \{ x \mid x \in S_2 \wedge x > 0 \}\\
		S_4 &=& \{x - 2 \mid x \in S_3 \}\\
		S_5 &=& \{x \mid x \in S_2 \wedge x \leq 0\}
		\end{array}
	\]
	}
	\only<2>{%
	\[
	\begin{array}{rcl}
		S_1 &=& \top\\
		S_2 &=& \{66666, 66664, \ldots, 2, 0\}\\
		S_3 &=& \{66666, 66664, \ldots, 4, 2\}\\
		S_4 &=& \{66664, 66662, \ldots, 2, 0\}\\
		S_5 &=& \{0\}\\
	\end{array}
	\]
	}
	\end{overlayarea}
	\vfill
\end{frame}

\subsection{Abstract Interpretation with Intervals}

\begin{frame}[fragile]
	\frametitle{\insertsubsection}

	\vspace{-1cm}
	\begin{center}
	\begin{minipage}{.5\textwidth}
	\begin{tabbing}
		1234\=1234\=1234\=\kill
		1:\>$x \leftarrow 66666$\\
		2:\>while $x > 0$ do\\
		3:\>\>x $\leftarrow x - 2$\\
		4:\>done\\
		5:\>
	\end{tabbing}
	\end{minipage}\qquad\qquad~~
	\begin{minipage}{.5\textwidth}
		\tikzstyle{nnn}=[circle]
		\begin{tikzpicture}[auto,node distance=1.3cm, thick]
			\node[nnn] (n1) {1};
			\node[nnn] (n2) [below of=n1] {2};
			\node[nnn] (n3) [below of=n2] {3};
			\node[nnn] (n4) [below of=n3] {4};
			\node[nnn] (n5) [below left of=n2] {5};
			\draw [->] (n1) -- (n2);
			\draw [->] (n2) -- (n3);
			\draw [->] (n3) -- (n4);
			\draw [->] (n2) -- (n5);
			\draw [->] (n4) to [out=0, in=0] (n2);
		\end{tikzpicture}
	\end{minipage}
	\end{center}
	\vfill

	\begin{overlayarea}{\textwidth}{0cm}
		\vspace{-1cm}
	\only<1>{%
	\[
		\begin{array}{rcl}
		S'_1 &=& \top\\
		S'_2 &=& [66666, 66666] \sqcup S'_4\\
		S'_3 &=& S'_2 \sqcap [1, +\infty]\\
		S'_4 &=& S'_3 - [2,2]\\
		S'_5 &=& S'_2 \sqcap [-\infty, 0]\\
		\end{array}
	\]
	}
	\only<2>{%
	\[
	\begin{array}{rcl}
		S_1 &=& \top\\
		S_2 &=& [-1, 66666]\\
		S_3 &=& [1, 66666]\\
		S_4 &=& [-1, 66664]\\
		S_5 &=& [-1, 0]
	\end{array}
	\]
	}
	\end{overlayarea}
	\vfill
\end{frame}


\subsection{So What's the Problem?}
\begin{frame}[fragile]
	\frametitle{\insertsubsection}

	\begin{itemize}
		\item The control flow graph (CFG) is required.
		\item Indirect jumps in binary code make the CFG hard to extract.
	\end{itemize}

	\vfill

\begin{lstlisting}
mov eax, [ebp-0x8]  ; eax := *(ebp - 8)
sub eax, 0x2        ; eax := eax - 2
cmp eax, 0x5        ; cf := (0 =< eax < 5)
                    ; zf := (eax = 5)
ja  0xd8            ; jump if cf = 0 and zf = 0
jmp [0x8048a0c + eax*4]
\end{lstlisting}
\vfill

	\begin{itemize}
	\item Need to know the possible values of \texttt{eax} at the \texttt{jmp}.
	\item Can't get the register values without the CFG.
	\item Can't get CFG without the register values.
	\end{itemize}
\vfill
\end{frame}

\end{document}
